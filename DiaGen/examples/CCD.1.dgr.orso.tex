\documentclass[letterpaper,10pt,fleqn,leqno,onecolumn]{article}
\usepackage[margin=2.0cm]{geometry}
\title{Got Tensor Contractions?}
\author{Dmitry I. Lyakh}
\begin{document}
\maketitle
\section{DEFINITIONS}
 1. The values of consecutive indices which are not separated by commas
    are assumed to form an ascending sequence (partially symmetrical storage). \newline
 2. Index groups separated by commas run independently. \newline
 3. Index ranges: \newline
    m,n,l: full occupied orbital range (e.g., [0..99]); \newline
    e,f,d: full virtual orbital range (e.g., [100..999]); \newline
    I,J,K: a small (active) subrange of the occupied orbital range (e.g., [90..99]); \newline
    A,B,C: a small (active) subrange of the virtual orbital range (e.g., [990..999]); \newline
    i,j,k: the occupied orbital range excluding the active subrange (e.g., [0..89]); \newline
    a,b,c: the virtual orbital range excluding the active subrange (e.g., [100..989]). \newline
 4. Despite multiple forms in which S/L/C tensors may appear in contractions,
    the only unique elements of these tensors are those which have both upper
    and lower indices ordered in an ascending order (separately). \newline
\section{TENSOR EXPRESSIONS}
%#CCD: Energy
\begin{equation} \;\;\;\;\;\;  2: Z+=H^{l_{1}l_{2}}_{d_{1}d_{2}}S^{d_{1}d_{2}}_{l_{1}l_{2}} \end{equation}
\begin{equation} \;\;\;\;\;\;  1: Z^{e_{1}e_{2}}_{m_{1}m_{2}}+=H^{e_{1}e_{2}}_{m_{1}m_{2}} \end{equation}
\begin{equation} \;\;\;\;\;\;  3: Z^{e_{1}e_{2}}_{m_{1}m_{2}}+=H^{e_{1}}_{d_{1}}S^{e_{2},d_{1}}_{m_{1}m_{2}}\cdot -1/1 \end{equation}
\begin{equation} \;\;\;\;\;\;  4: Z^{e_{1}e_{2}}_{m_{1}m_{2}}+=H^{l_{1}}_{m_{1}}S^{e_{1}e_{2}}_{m_{2},l_{1}} \end{equation}
\begin{equation} \;\;\;\;\;\;  5: Z^{e_{1}e_{2}}_{m_{1}m_{2}}+=H^{e_{1}e_{2}}_{d_{1}d_{2}}S^{d_{1}d_{2}}_{m_{1}m_{2}} \end{equation}
\begin{equation} \;\;\;\;\;\;  6: Z^{e_{1}e_{2}}_{m_{1}m_{2}}+=H^{e_{1},l_{1}}_{m_{1},d_{1}}S^{e_{2},d_{1}}_{m_{2},l_{1}} \end{equation}
\begin{equation} \;\;\;\;\;\;  7: Z^{e_{1}e_{2}}_{m_{1}m_{2}}+=H^{l_{1}l_{2}}_{m_{1}m_{2}}S^{e_{1}e_{2}}_{l_{1}l_{2}} \end{equation}
\begin{equation} \;\;\;\;\;\;  8: Z^{e_{1}e_{2}}_{m_{1}m_{2}}+=H^{l_{1},l_{2}}_{d_{1}d_{2}}S^{e_{1}e_{2}}_{m_{1},l_{1}}S^{d_{1}d_{2}}_{m_{2},l_{2}}\cdot -1/1 \end{equation}
\begin{equation} \;\;\;\;\;\;  9: Z^{e_{1}e_{2}}_{m_{1}m_{2}}+=H^{l_{1}l_{2}}_{d_{1}d_{2}}S^{e_{1}e_{2}}_{l_{1}l_{2}}S^{d_{1}d_{2}}_{m_{1}m_{2}} \end{equation}
\begin{equation} \;\;\;\;\;\;  10: Z^{e_{1}e_{2}}_{m_{1}m_{2}}+=H^{l_{1}l_{2}}_{d_{1},d_{2}}S^{e_{1},d_{1}}_{m_{1}m_{2}}S^{e_{2},d_{2}}_{l_{1}l_{2}}\cdot -1/1 \end{equation}
\begin{equation} \;\;\;\;\;\;  11: Z^{e_{1}e_{2}}_{m_{1}m_{2}}+=H^{l_{1},l_{2}}_{d_{1},d_{2}}S^{e_{1},d_{1}}_{m_{1},l_{1}}S^{e_{2},d_{2}}_{m_{2},l_{2}}\cdot 1/2 \end{equation}
\end{document}
